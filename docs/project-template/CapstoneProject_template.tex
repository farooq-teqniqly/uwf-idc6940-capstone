\documentclass[12pt, a4paper,twoside]{report}

%% Every LaTeX document begins with a preamble, which loads packages and
%% defines various settings to make the document look right. Mostly,
%% you can ignore everything in this template before \begin{document} on
%% line 74

\usepackage{mathtools,amsthm} % Enable useful mathematical symbols/environments
\usepackage{graphicx} % Enable graphics
\usepackage{fancyhdr,titlesec,microtype} % enable various formatting commands
\usepackage[margin=2.5cm]{geometry} % Set margin size
\usepackage{palatino} % Set the font
\usepackage[latin1]{inputenc} % Allow you to input accents, umlauts and other characters
\usepackage[T1]{fontenc} % Lets LaTeX print a wider array of characters

\usepackage{xcolor} % Enable coloured elements
\definecolor{mypurple}{HTML}{004C97} %%% Purple
\definecolor{myred}{HTML}{004C97} %%%EssexOrange
\definecolor{myblue}{HTML}{004C97} %%%Seagrass

% For technical reasons, hyperref should be loaded after all other packages
\usepackage[colorlinks,linkcolor=myblue,citecolor=mypurple]{hyperref}

\renewcommand{\baselinestretch}{1.5} % 1.5 line spacing

% Define \begin{theorem}, \end{theorem}, etc.
\theoremstyle{plain} % The following will be italicised
\newtheorem{theorem}{Theorem}[chapter]
\newtheorem{lemma}[theorem]{Lemma}
\newtheorem{proposition}[theorem]{Proposition}
\newtheorem{corollary}[theorem]{Corollary}

\theoremstyle{definition} % The following environments will not use italics
\newtheorem{definition}[theorem]{Definition}
\newtheorem{example}[theorem]{Example}

\theoremstyle{remark} % The following environments will not use italics or bold titles
\newtheorem{remark}[theorem]{Remark}

\numberwithin{equation}{chapter}

% Fancy headings
\pagestyle{fancy}
\setlength{\headheight}{15pt}
\fancyheadoffset[LE,RO]{0pt}
\renewcommand{\chaptermark}[1]{\markboth{#1}{}}
\renewcommand{\sectionmark}[1]{\markright{\thesection\ #1}}
\fancyhf{}
\fancyhead[LE]{\makebox[0pt][l]{\thepage}\hfill\leftmark}
\fancyhead[RO]{\rightmark\hfill\makebox[0pt][r]{\thepage}}
\fancypagestyle{plain}{%
    \fancyhead{} % get rid of headers
    \renewcommand{\headrulewidth}{0pt} % and the line
}

% Fancy chapter numbers
\titleformat{\chapter}[display]
    {\normalfont\bfseries\color{myred}}
    {\filleft\hspace*{-60pt}%
        \rotatebox[origin=c]{90}{%
            \normalfont\color{black}\Large%
            \textls[180]{\textsc{\chaptertitlename}}%
        }
        \hspace{10pt}%
        {\setlength\fboxsep{0pt}%
            \colorbox{myred}{\parbox[c][3cm][c]{2.5cm}{%
                \centering\color{white}\fontsize{80}{90}\selectfont\thechapter}%
            }
        }
    }
    {10pt}
    {\titlerule[2.5pt]\vskip3pt\titlerule\vskip4pt\LARGE\sffamily}

\begin{document} % Start your document

%%%%%%%%%%%% BEGIN TITLE PAGE %%%%%%%%%%%%

\thispagestyle{empty} % For the title page, no header / footer

\noindent
    \begin{minipage}{0.1\textwidth}
    \includegraphics[height=4.5em]{UWF.png}
    \end{minipage}
    \hfill
    \begin{minipage}{0.5\textwidth}
    \begin{center}
        \renewcommand\familydefault{\sfdefault}
        \fontfamily{phv}\selectfont
        {\large Departmemt of Mathematics and Statistics}
    \end{center}
    \end{minipage}

\begin{center}
    \noindent\textcolor{myred}{\rule{\linewidth}{4.8pt}}
    
    \vspace{2em}
    \noindent {\LARGE \sc Capstone Project Dissertation}
    
    \vspace{3em}
    \noindent {\Huge{\color{myblue} YOUR PROJECT TITLE HERE}}
    
    \vspace{3em}
    \noindent {\Large \bf YOUR NAME HERE}
    \vfill
    \noindent {\Large {Supervisor:} {\color{mypurple} \bf YOUR SUPERVISOR NAME HERE}}
    
    \vspace{0.5em}
    \noindent\textcolor{myred}{\rule{\linewidth}{4.8pt}}
    
    \vspace{2em}
    {\Large \today }
    
    {\Large }
\end{center}

\clearpage

%%%%%%%%%%%% END TITLE PAGE %%%%%%%%%%%%

\tableofcontents

% If you have lots of figures with captions / numbers, uncomment the following line
% \listoffigures

% If you have lots of tables and want a list of them, uncomment the following line
% \listoftables

\chapter{Introduction}\label{ch:1}

The first chapter of a Capstone document is usually an overview of what you've done. Often, it is the last chapter you will write.

\begin{equation}
    f(x)=2\pi+\alpha
\end{equation}

\begin{theorem} \label{example-theorem}
Sometimes, you will want to state the main results of your document in the introduction.
\end{theorem}

\begin{remark}
LaTeX is clever, and automatically generates numbers for theorems, remarks and anything else you might want to label. You can give these an invisible name using \verb!\label{your-key}! and referring back to it later using \verb!\ref{your-key}!, for example the following number will be the same as the theorem above: Theorem \ref{example-theorem}.
\end{remark}

Similarly, you will want to reference external sources as you write your document. The basic way to do this is to add \verb!\bibitem{your-chosen-key}!s at the end of your document (this template has three examples), and use \verb!\cite{your-chosen-key}! to refer to it. For instance, if I wanted to cite the example document by Noether, I can write \cite{bookByTuring}.

Mathematics is added using dollar signs for in-line math, i.e. $x^2 + y^2 = z^2$, or by using open-bracket close-bracket for a displayed equation.
\[ c^2 = a^2 + b^2 - 2ab\cos \theta. \]

Ordered lists are written using the \verb!enumerate! environment:
\begin{enumerate}
    \item Hello.
    \item This is the second item in my list.
\end{enumerate}

I can also write unordered lists using \verb!itemize!:
\begin{itemize}
    \item Hello.
    \item This is now the second item in my list.
\end{itemize}

\chapter{Your first main chapter}\label{ch:2}

Here is the meat of your document

\section{Your first section of the first main chapter}\label{sec:2.1}

... goes here.



\chapter{Conclusions}\label{ch:concl}

And here is the final chapter showing how clever you are ....



%%% Remove the following example bibliography items and add your own %%%

\begin{thebibliography}{999} % The '999' here tells LaTeX how much space to reserve for the bibliography item numbers. It has nothing to do with the actual number of references.

%%% Bibliography items should be below this here %%%

\bibitem{myFirstReference} % This is a key, so you can cite this with \cite{myFirstReference}
E.~Noether.
\newblock Invariante {V}ariationsprobleme.
\newblock {\em Nachr. d. K{\"o}nig. Gesellsch. d. Wiss. zu G{\"o}ttingen,
  Math-phys. Klasse, {Seite 235-157}}, 1918.

\bibitem{bookByTuring}
A.~M. Turing.
\newblock Computing machinery and intelligence.
\newblock {\em Mind}, 59:433--460, 1950.

\bibitem{myFakeBook}
J.~Fakename.
\newblock Name of book or article goes here.
\newblock \emph{Journal name}, page numbers, year, other specific info.

\end{thebibliography}

\end{document}

