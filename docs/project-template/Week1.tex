\documentclass[addpoints, answers]{exam}


%This is where you'll edit your class information for each test
\newcommand{\instructor}{\textbf{Instructor: Shusen Pu}}
\newcommand{\class}{\textbf{Capstone}}
\newcommand{\examnum}{\textbf{\, \, HW \#1}}
\newcommand{\examdate}{March 3, 2022}
\newcommand{\name}{\textbf{Student Name (print)}: \underline{\hspace{2in}}   }
\usepackage{listings}
\usepackage[shortlabels]{enumitem}
\usepackage{amssymb}
\usepackage{mathtools}
\usepackage{amsthm}
\usepackage{amsmath}
\usepackage{cancel}
\usepackage{verbatim}
\usepackage{graphicx}
\usepackage{relsize}
\usepackage{marvosym}
\usepackage{dirtytalk}
\usepackage{wrapfig}
\usepackage{hyperref}
\hypersetup{colorlinks=true,urlcolor=blue}
\usepackage{upgreek}
\usepackage{graphicx}
\usepackage{color}
\newcommand{\ra}{\rightarrow}
\newcommand{\la}{\leftarrow}
\newcommand{\suchthat}{\textnormal{ such that }}
\newcommand{\for}{\textnormal{ for }}
\newcommand{\where}{\textnormal{ where }}
\newcommand{\by}{\textnormal{ by }}
\newcommand{\when}{\textnormal{ when }}
\newcommand{\then}{\textnormal{ then }}
\newcommand{\mn}{\medskip \noindent}

\newcommand{\ds}{\displaystyle}
%\usepackage{lipsum}


\newcommand\afterclasspart{\renewcommand\partlabel{(\thepartno)\makebox[0pt]{$\ ^\star$}}}  
\newcommand\standardpart{\renewcommand\partlabel{(\thepartno)}}
\newcommand\afterclassquestion{\renewcommand\questionlabel{\makebox[2pt]{$\bigstar$}\, \thequestion.}}  
\newcommand\afterclassquestions{\renewcommand\questionlabel{\makebox[4pt]{$\bigstar\bigstar$}\, \,  \thequestion.}} 

\newcommand\standardquestion{\renewcommand\questionlabel{\thequestion.}} 

%\newcommand\standardquestion{\renewcommand\questionlabel{\thequestion.}} 

\newenvironment{modenumerate}
  {\enumerate\setupmodenumerate}
  {\endenumerate}

\newif\ifmoditem
\newcommand{\setupmodenumerate}{%
  \global\moditemfalse
  \let\origmakelabel\makelabel
  \def\moditem##1{\global\moditemtrue\def\mesymbol{##1}\item}%
  \def\makelabel##1{%
    \origmakelabel{\ifmoditem\llap{\mesymbol\enspace}\fi##1}%
    \global\moditemfalse}%
}

\usepackage{pgfplots}
\pgfplotsset{compat=1.11}
\newcommand{\e}{\mathrm{e}}

\newcommand{\R}{\mathbb{R}}
\newcommand{\Z}{\mathbb{Z}}
\newcommand{\N}{\mathbb{N}}
\newcommand{\Q}{\mathbb{Q}}
\newcommand{\M}{\mathbb{M}}
\newcommand{\C}{\mathbb{C}}

\renewcommand{\P}{\mathcal{P}}
\newcommand{\T}{\mathcal{T}}
\newcommand{\D}{\mathcal{D}}
\newcommand{\U}{\mathcal{U}}
\newcommand{\V}{\mathcal{V}}

\newcommand{\Rn}{\R^n}
\newcommand{\Mn}{\M_n}
\newcommand{\Mmn}{\M_{m,n}}
\newcommand{\seq}[1]{\{#1_i\}_{i=1}^{\infty}}
\newcommand{\trace}{\mathrm{tr}}
\newcommand{\rsa}{\rightsquigarrow}
\newcommand{\rank}{\mathrm{rank}}
\newcommand{\proj}{\mathrm{proj}}
\newcommand{\vol}{\mathrm{vol}}

\newcommand{\mat}[1]{\begin{bmatrix} #1 \end{bmatrix}}
\DeclarePairedDelimiter\ceil{\lceil}{\rceil}
\DeclarePairedDelimiter\floor{\lfloor}{\rfloor}


%This is where you can print out your solutions

\printanswers
%\noprintanswers

\begin{document}



% This is where you change the header of the first page and the running header
\pagestyle{headandfoot} 
\firstpageheader{\Large{{\class}}}{\Large{\examnum}}{\instructor}
\runningheader{\class\ \examnum}{Page \thepage\ of \numpages}{}
\runningheadrule
%\runningfooter{Page \thepage\ of \numpages}{}{\hfill{\large{\textbf{Points earned: \makebox[.5in]{\hrulefill} out of \pointsonpage{\thepage} points}}}}	
%\runningfooter{Page \thepage\ of \numpages}{}{}	
%\vspace{0.5cm}		
\name\\ [0.4cm]
This assignment contains \numpages\ pages and is due at \textbf{11:59 pm on Sunday.}\\

{\begin{center}
   {\gradetable[h]} 
\end{center}}

You can learn how to insert images into Overleaf at: \url{https://www.overleaf.com/learn/latex/Inserting_Images}. Two example codes are commented using $\%$ in question 1.
\begin{questions}
 \question [2] 
 \textbf{Introduce yourself and meet your classmates:}
 Please upload/insert a screenshot here.

% Two examples are provided below:

%\includegraphics[scale=0.1]{UWF.png}

% uncomment to use the following code
%\begin{figure}[htbp!]
   % \centering
    %\includegraphics[width=0.49\textwidth]{UWF.png}
    %\caption{xxx}
    %\label{fig:uwf}
%\end{figure}

 \vfill

\question [2] 
\textbf{Join the course Discord channel.} 
 Please upload/insert a screenshot here.
\vfill
\newpage

\question[2] List a tentative topic or method that you’re interested in exploring during this course. You have the flexibility to change it up until week 5.
\vfill

\question[5] Add your name and research/methods topics in the Google document and share a screenshot here.

\href{https://docs.google.com/document/d/1PAWyJc56pf78WBtep3CvAGaMZZjem8gUJfEB1aeCQ7A/edit?usp=sharing}{https://docs.google.com/document/d/1PAWyJc56pf78WBtep3CvAGaMZZjem8gUJfEB1aeCQ7A/edit?usp=sharing}
 
\vfill

\question[5] Please review and study the following two methods for presenting your progress and research report:

\begin{enumerate}
\item Write your report on Overleaf. You can copy or download the template from: 
\href{https://www.overleaf.com/read/tgrddxgwrhst\#ff33cc}{\url{https://www.overleaf.com/read/tgrddxgwrhst\#ff33cc}}.

\item Write your report on GitHub. The template is available here:
\href{https://github.com/capstone4ds/capstone4ds_template}{\url{https://github.com/capstone4ds/capstone4ds_template}}.
You can find instructions for setting up the website on the same page.
\end{enumerate}


Please indicate which method you will use for this course.

\vfill


\end{questions}


\end{document}
