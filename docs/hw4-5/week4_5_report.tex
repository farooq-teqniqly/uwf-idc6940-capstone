\documentclass[12pt, a4paper]{article}
\usepackage[margin=2.5cm]{geometry}
\usepackage{graphicx}
\usepackage{amsmath}
\usepackage{booktabs}
\usepackage{parskip}
\usepackage[T1]{fontenc}
\usepackage[utf8]{inputenc}
\usepackage[colorlinks,linkcolor=blue,citecolor=blue]{hyperref}

\graphicspath{{./images/}}

\title{Week 4--5 Progress Report: Methods, Data, and Implementation}
\author{Farooq Mahmud}
\date{\today}

\begin{document}
\maketitle
\tableofcontents
\clearpage

%=============================================================================
\section{Project Goal and Research Question}
%=============================================================================

\subsection{Goal}
% TODO: State the capstone project goal (trip duration forecasting for NYC Uber daily averages; compare time-series vs LSTM).
\ldots

\subsection{Research Question}
% TODO: State the research question (e.g., How do ARIMA and LSTM compare for 14-day forecasting of daily average trip duration on sMAPE and MASE?).
\ldots

%=============================================================================
\section{Expanded Methods}
%=============================================================================

\subsection{Problem Definition}
% TODO: What you are predicting/estimating: daily average Uber trip duration (minutes) in NYC. Inputs: univariate time series (366 days). Outputs: 14-day point forecast; for comparison, sMAPE and MASE.
\ldots

\subsection{Data Description}
% TODO: Data source (NYC TLC FHV); key variables (pickup\_date, avg\_duration\_min); expected sample size (366 days). Preprocessing: PySpark aggregation to daily averages (include short excerpt in narrative; do not cite prior report).
\ldots

\subsection{Preprocessing Plan}
% TODO: Cleaning, missing data, scaling, train/validation/test split (chronological; last 14 days test), sequence construction for LSTM (sliding window).
\ldots

\subsection{Modeling Plan}
% TODO: (1) ARIMA: include relevant excerpts (ARIMA(3,1,2), 14-day forecast, AIC/BIC, Ljung--Box) in narrative; do not cite prior report. (2) LSTM: R/keras, one LSTM layer, sequence length, recursive 1-step 14-day forecast. (3) Comparison: same horizon, same metrics (sMAPE, MASE).
\ldots

\subsection{Evaluation Plan}
% TODO: Metrics: sMAPE and MASE over the 14-day forecast. MASE scaled by naive one-step MAE on training data. Same test period for both methods. Table and figure comparing ARIMA vs LSTM.
\ldots

%=============================================================================
\section{Dataset and Access}
%=============================================================================

\subsection{Source}
% TODO: NYC TLC High-Volume FHV data \cite{nyctlc2024}. Analysis-ready series: finalproject.csv (daily average trip duration for 2024). Describe data production (e.g.\ PySpark aggregation) in narrative using a short excerpt; do not cite prior report.
\ldots

\subsection{Access and Loading}
% TODO: Dataset citation/link; confirm accessible. Proof of load: dimensions (366 rows $\times$ 2 columns), variable names and types, missing values (if any). Code: lstm\_forecast.qmd or load script.
\ldots

%=============================================================================
\section{Implementation and Experiments}
%=============================================================================

\subsection{What Was Implemented}
% TODO: (1) ARIMA: reproduce 14-day forecast using existing R/Quarto code; include excerpts of method and results in this report (do not cite prior report). (2) LSTM: run lstm\_forecast.qmd (R/keras) for 14-day forecast and sMAPE/MASE. (3) Comparison: script to combine ARIMA and LSTM forecasts and compute metrics.
\ldots

\subsection{What Was Tested}
% TODO: Dataset (finalproject.csv), subset (full 2024; test = last 14 days for LSTM comparison), variables (avg\_duration\_min).
\ldots

\subsection{What Worked and What Did Not}
% TODO: Bugs, limitations, unexpected issues (e.g., small sample size for LSTM, overfitting, hyperparameters).
\ldots

\subsection{Preliminary Outputs}
% TODO: ARIMA 14-day forecast table and plot (included as excerpts in this report). LSTM forecast figure and training loss (from lstm\_forecast.qmd). Comparison table (ARIMA vs LSTM sMAPE, MASE) and plot.
\ldots

%=============================================================================
\section{Results (Preliminary)}
%=============================================================================

\subsection{ARIMA}
% TODO: 14-day forecast summary; include excerpt(s) of forecast table and figure in the narrative (do not cite prior report).
\ldots

\subsection{LSTM}
% TODO: 14-day forecast; training/validation loss curve; sMAPE and MASE. Figures from lstm\_forecast.qmd.
\ldots

\subsection{Comparison (ARIMA vs LSTM)}
% TODO: Table: Model | sMAPE (\%) | MASE. Figure: actual vs ARIMA vs LSTM over the 14 days. Short interpretation: which method performed better and why.
\ldots

%=============================================================================
\section{Issues and Limitations}
%=============================================================================
% TODO: Small sample size (366 points) for LSTM; univariate models; any bugs or data issues encountered. Note limitations in the narrative; do not cite prior report.
\ldots

%=============================================================================
\section{Next Steps}
%=============================================================================
% TODO: Complete comparison script; add ARIMA 14-day forecast to comparison if not yet done; extend with more years of data if available; refine LSTM hyperparameters; finalize report and submit PDF.
\ldots

%=============================================================================
\bibliographystyle{plain}
\bibliography{week4_5_report}
%=============================================================================
\end{document}
